\section{Modelling Assumptions}
Usually, a system behaviour is seen as a sequence of outcomes of an observation for the system.
Each of these outcomes is called a message.
It is natural to assume that each outcome is caused by that in the system and in its environment some complex of events has happened.
Further, we should note that the set of events that we observe depends essentially on the aim of our observation.
This aim is determined by the focus of our interest under studying the system.
Therefore, we should assume that the set of observed events is predefined.

Thus, we can formulate the following modelling assumptions.
\begin{assumption}\label{ma:1}
A system is described by a finite set of event sources.
These event sources are below called {\bfseries clocks}.
\end{assumption}
\begin{assumption}\label{ma:2}
Event instances having the same source are similar and ordered linearly by them occurrences.
In other words, each event occurrence is identified by its source and its number of occurrence in the series of occurrences of events having the same source.
An event occurrence generated by a clock is below called a {\bfseries tick} of this clock.
\end{assumption}
\begin{assumption}\label{ma:3}
A {\bfseries message} is a reference to a subset of the set of clocks.
It contains references to those clocks that have generated ticks described by this message.
\end{assumption}
\begin{assumption}\label{ma:4}
Each individual system behaviour is specified as a sequence of messages.
Such a sequence of messages we call bellow a {\bfseries schedule}.
\end{assumption}
\begin{assumption}
In general, a system {\bfseries behaviour} is a set of individual system behaviours that are admissible in accordance with the specification of system requirements.
\end{assumption}
Thus, the phrase ``a schedule represents an admissible individual system behaviour'' means that this schedule belongs to the set of schedules that represents the system behaviour or, in other words, this schedule holds the property ``to be admissible''.
These reasoning lead to the general concept of property, namely, a property is a set of schedules.
\begin{assumption}
A {\bfseries property} is a set of schedules.
\end{assumption}
\begin{assumption}
A {\bfseries system satisfies the property} in the case where each admissible schedule of the system belongs to the set that corresponds to this property.
\end{assumption}
Now we explain informally the meaning of the term a safety property.
\begin{assumption}
A property is a safety property if we have a way to determine after some final sequence of observations whether this property holds for a schedule or not.
\end{assumption}

